\documentclass{article}

\usepackage{fancyhdr} % Required for custom headers
\usepackage{lastpage} % Required to determine the last page for the footer
\usepackage{extramarks} % Required for headers and footers
\usepackage[usenames,dvipsnames]{color} % Required for custom colors
%\usepackage{courier} % Required for the courier font
\usepackage{mathtools}
\usepackage{amssymb}
\usepackage{multicol}

\usepackage{paratype}
\renewcommand*\familydefault{\sfdefault} %% Only if the base font of the document is to be sans serif
\usepackage[T1]{fontenc}

% Margins
\topmargin=-0.45in
\leftmargin=0.45in
\rightmargin=0.45in
\evensidemargin=0in
\oddsidemargin=0in
\textwidth=6.5in
\textheight=9.0in
\headsep=0.25in

\linespread{1} % Line spacing

% Set up the header and footer
\pagestyle{fancy}
\lhead{\authorName} % Top left header
\chead{\class\ : \Title} % Top center head
\rhead{} % Top right header
\lfoot{Typed by JoshAshby.\\Got questions or corrections?} % Bottom left footer
\cfoot{\authorContact} % Bottom center footer
\rfoot{Page\ \thepage\ of\ \protect\pageref{LastPage}} % Bottom right footer
\renewcommand\headrulewidth{0pt} % Size of the header rule
\renewcommand\footrulewidth{0pt} % Size of the footer rule
\setlength{\columnseprule}{1pt}
\setlength\parindent{0pt} % Removes all indentation from paragraphs

\newcommand{\Title}{Various key notes} % Assignment title
\newcommand{\class}{Calc} % Course/class
\newcommand{\authorName}{\colorbox{BrickRed}{\color{White}{JoshAshby}}}
\newcommand{\authorContact}{{\color{BrickRed}[}
joshuaashby@joshashby.com{\color{BrickRed}]}\\
{\color{BrickRed}[}(720) - 663 -1279{\color{BrickRed}]}}

\begin{document}
\(<5,\sqrt{6.7}i>\) - Stand back, I know \(i\)Vectors!\\
\section*{\color{Gray}Convergence and Infinite Series}
  \subsection*{\color{BrickRed}Series}
  \begin{itemize}
    \item Geometric series\\
      \subitem \( \sum ar^n \) \\
      if \( |r|<1 \) then \( \sum ar^n \) converges\\
      The sum can be found with \(\frac{a}{1-r}\)\\
      \subitem eg:\\
      \( \sum\limits_{n=0}^{\infty} \frac {4} {2^n} =
      \sum\limits_{n=0}^{\infty} 4 \times (\frac {1} {2})^n \) where
      \(a=4\) and \(r=\frac {1} {2} < 1 \therefore \sum\limits_{n=0}^{\infty}
      \frac{4}{2^n} \) converges \\
    \item P-Series
      \subitem \( \sum \frac{1}{n^p}\) where
      \( p>0 \) \\
      if \(p>1\) then the series converges\\
      if \(p<=1\) then the series diverges\\
      if \(p=1\) then the series is harmonic and diverges (which is
      useful for the comparison test)\\
    \item Alternating Series
      \subitem \( \sum (-1)^n a_n \)\\
      converges if \( \lim_{n\rightarrow\infty} a_n=0 \) and \( a_{n+1}<a_n \)\\
  \end{itemize}
  \subsection*{\color{BrickRed}Tests}
  \begin{itemize}
    \item $n^{th}$ term test - Only for divergence\\
      \subitem if \( \lim_{n\rightarrow\infty}a_n\neq0 \) then \( \sum
      a_n\) diverges\\
    \item Integral Test\\
      \subitem if \( \int\limits_{1}^{\infty} f(x) \, dx \) converges
      then \( \sum\limits_{n=1}^{\infty} a_n\) converges where \(
      f(x)=a_n\). The converse of this is also true.\\
    \item Limit Comparison\\
      \subitem if \( \lim_{n\rightarrow\infty} \frac{a_n}{b_n} = L \)
      where L is both positive and finite, then the two series
      both either converge or diverge.\\
      \subitem eg:\\
      \( \sum\limits_{n=3}^{\infty} \frac{3}{\sqrt{n^2-4}} \)
      compared to 
      \( \sum\limits_{n=1}^{\infty}\frac{1}{n} \)\\
      \( \lim_{n\rightarrow\infty}\left( \frac{3}{\sqrt{n^2-4}}
      \times \frac{n}{1} \right)=\frac{\infty}{\infty}\)\\
      \(L'H \rightarrow \lim_{n\rightarrow\infty}\left(
      \frac{3\sqrt{n^2-4}}{n} \right)=\frac{\infty}{\infty}\)\\
      \(\lim_{n\rightarrow\infty}\left(
      \frac{3\sqrt{n^2-4}}{\sqrt{n^2}} \right) =
      3\lim_{n\rightarrow\infty}\sqrt{\frac{n^2-4}{n^2}} =
      3\sqrt{\lim_{n\rightarrow\infty}\frac{n^2-4}{n^2}} =
      3 \times 1 = 3 \therefore \sum\limits_{n=0}^{\infty}
      \frac{3}{\sqrt{n^2-4}}\) diverges since \( \frac{1}{n} \)
      diverges and the limit is finite and positive.\\
    \item Ratio Test
      \subitem \(\lim_{n\rightarrow\infty}\left|
      \frac{a_{n+1}}{a_n} \right|\)\\
      if \(<1 ~ a_n\) converges\\
      if \(>1 ~ a_n\) diverges\\
      if \(=1\) the test is inconclusive\\
      \subitem eg:\\
      Find the range of x where
      \(\sum\limits_{n=3}^{\infty}\frac{(-1)^nn!(x-4)^n}{3^n}\)
      converges.\\
      \(\lim_{n\rightarrow\infty}\left|\frac{(n+1)!(x-4)^{n+1}}{3^{n+1}}
      \times \frac{3^n}{n!(x-4)^n}\right|<1\)\\
      \( \lim_{n\rightarrow\infty}\left| \frac{(x-4)(n+1)}{3}
      \right|<1\) \\
      \( \left| \frac{x-4}{3} \right|
      \lim_{n\rightarrow\infty} \left| n+1 \right|<1 \therefore \)
      no range of x makes \(
      \sum\limits_{n=0}^{\infty}\frac{(-1)^nn!(x-4)^n}{3^n}\)
      converge.\\
    \item Condition Convergence\\
      \subitem if \( \sum\limits_{n=1}^{\infty} a_n \) converges
      and \( \sum\limits_{n=1}^{\infty} \left| a_n \right| \)
      diverges\\
  \end{itemize}
\section*{\color{Gray}Taylor Series}
  \begin{multicols}{2}
  \begin{itemize}
    \item \(
      f(x)=\sum\limits_{n=0}^{\infty}\frac{f^n(c)}{n!}(x-c)^n\)\\
      if \(c=0\) then the series is a MacLaurin series\\
      \subitem
      \(f(x)=\sum\limits_{n=0}^{\infty}\frac{f^n(0)}{n!}x^n\)\\
    \item  \(\sin(x)=\sum\limits_{n=0}^{\infty} (-1)^n
      \frac{x^{2n+1}}{(2n+1)!}\)\\
    \item \(\cos(x)=\sum\limits_{n=0}^{\infty}(-1)^n\frac{x^{2n}}{(2n)!}
     \)\\
    \item\( e^x=\sum\limits_{n=0}^{\infty}\frac{x^n}{n!} \)\\
      \subitem \( e^{-x}=\sum\limits_{n=0}^{\infty}(-1)^n
      \frac{x^n}{n!} \)\\
    \item \(\ln(x+1)=\sum\limits_{n=0}^{\infty}(-1)^n\frac{x^n}{n}\)\\
      \columnbreak
    \item Example of finding the Taylor series centered at 0 (aka
      MacLaurin) of \(e^x\)\\
      \subitem \(f(x)=e^x\)\\
      \subitem \(f(0)=1\)\\
      \(f'(x)=e^x\) and \(f'(0)=1\)\\
      \(f''(x)=e^x\) and \(f''(0)=1\)\\
      \(f'''(x)=e^x\) and \(f'''(0)=1\)\\
      \(\frac{1}{0!}x^0+\frac{1}{1!}x^1+\frac{1}{2!}x^2+\frac{1}{3!}x^3+\cdots+\frac{1}{n!}x^n\)\\
      \(\therefore f(x)=e^x=\sum\limits_{n=0}^{\infty}
      \frac{x^n}{n!}\)\\
  \end{itemize}
\end{multicols}
\newpage
\section*{\color{Gray}Parametric and Polar Equations and Vectors}
  \subsection*{\color{BrickRed}Parametric}
  \begin{itemize}
   \Large \item \( \dot{x}=\frac{dx}{dt}\)\\
   \item \( \dot{y}=\frac{dy}{dt}\)\\
   \item
     \(\frac{dy}{dx}=\frac{\frac{dy}{dt}}{\frac{dx}{dt}}=\frac{\dot{y}}{\dot{x}}\) 
      and \huge
     \( \frac{d^2y}{\left(dx\right)^2} = \frac{\frac{d}{dt}\frac{dy}{dx}}{\frac{dx}{dt}} \)\\
   \normalsize
   \item \(Arc ~ length = \int\sqrt{\dot{x}^2+\dot{y}^2} \, dt\)\\
   \item \(Area/integration = \int y(t) \frac{dx}{dt} \, dt = \int y \dot{x} \, dt \)\\
   \item \(x=r\cos(\theta)\) and \(y=r\sin(\theta)\)\\
  \end{itemize}
  \subsection*{\color{BrickRed}Polar}
  \begin{itemize}
   \item \(r=\sqrt{x^2+y^2}\)\\
   \item \( r^2=x^2+y^2 \)\\
   \item If going from parametric to polar, you have to convert
      \(t\) to \(\theta\): \(\tan(\theta)=\frac{y}{x}\)\\
   \item \(Arc ~ length = \int \sqrt{r^2+\left( \frac{dr}{d\theta}
     \right)^2} \, d\theta\)
   \item \(Area/integration = \frac{1}{2} \int r^2 \, d\theta \)\\
  \end{itemize}
  \subsection*{\color{BrickRed}Vectors}
  \begin{itemize}
    \item To get it out of the way: You can't divide a
      vector by another vector, only by a scalar.\ldots\\
    \item Second of all, vectors are just parametric equations in
      disguse, they just have an x and a y componet represented
      by:\\
    \subitem $\vec{i}=x$ the x componet of a vector\\
    \subitem $\vec{j}=y$ the y componet of a vector\\
    \item All vectors have Magnitude and Direction.\\
      \subitem $ \vec{r}=x\vec{i}+y\vec{j} $\\
      \subitem $ \vec{r}=<x,y>$\\
        \subsubitem The magnitude (a unit vector) is the same as the $|\vec{r}|=\sqrt{x^{2}+y^{2}}$\\
	\subsubitem The dirrection (another unit vector ) is defined by $\frac{\vec{r}}{|\vec{r}|}$\\
    \item Addition \& Subtraction\\
      \subitem $\vec{a}+\vec{b}=\vec{c}$\\
      $\vec{c}=(a_x+b_x)\vec{i}+(a_y+b_y)\vec{j}$\\
      \subitem Simply add the x componets together for the new
      vectors x componet, and add the y componets together for the new
      vectors componet. What this is essentially doing is placing the tail of
      one vector at the head of the other.\\
    \item Multiplication\\
      \subitem Scalar product\\
      	\subsubitem $6 \times \vec{a}=(6\times
	a_x)\vec{i}+(6\times a_y)\vec{j}$\\
        \subsubitem Vector multiplied by a single number
	(scalar). This is also how you divide a vector by a
	scalar.\\
        \subsubitem Simply multiply the scalar out to both the x
	and y componets.\\
      \subitem Dot product\\
      \subsubitem $\vec{a}\cdot \vec{b}=(a_x\times b_x)+(a_y
      \times b_y)$\\
        \subsubitem Two vectors multiplied together\\
	\subsubitem Multiply the x componets together and add them
	to the y componets multiplied together. This results in a
	scalar, not a vector.\\
      \subitem Cross Product\\
        \subsubitem Two vectors multiplied together\\
	\subsubitem Results in another vector.\\
    \item Helpful vectors and finding relationships between two
      vectors\\
      \subitem Angle between two vectors\\
        \subsubitem $\cos{\theta}=\frac{\vec{a}\cdot
	\vec{b}}{|\vec{a}||\vec{b}|}$\\
      \subitem Projection vectors\\
      \subsubitem $proj_{\vec{a}}\vec{b}=\frac{\vec{a}\cdot \vec{b}}{|\vec{b}|^2}\times
      \vec{b}$\\
      \subitem Normal vectors\\
      \subsubitem $norm_{\vec{a}}\vec{b}=\vec{a}-proj_{\vec{a}}\vec{b}$\\
  \end{itemize}
\end{document}
