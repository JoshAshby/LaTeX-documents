\section{\color{Gray}Infinite Series, Convergence tests, and Taylor Series}
  \subsection{\color{BrickRed}Infinite Series}
  \begin{itemize}
    \item Geometric series\\
      \begin{multicols}{2}
      \subitem \( \sum ar^n \) \\
      if \( |r|<1 \) then \( \sum ar^n \) converges\\
      The sum can be found with \(\frac{a}{1-r}\)\\
      \columnbreak
      \subitem eg:\\
      \( \sum\limits_{n=0}^{\infty} \frac {4} {2^n} =
      \sum\limits_{n=0}^{\infty} 4 \times (\frac {1} {2})^n \) where
      \(a=4\) and \(r=\frac {1} {2} < 1 \therefore \sum\limits_{n=0}^{\infty}
      \frac{4}{2^n} \) converges \\
    \end{multicols}
    \item P-Series
      \subitem \( \sum \frac{1}{n^p}\) where
      \( p>0 \) \\
      if \(p>1\) then the series converges\\
      if \(p<=1\) then the series diverges\\
      if \(p=1\) then the series is harmonic and diverges (which is
      useful for the comparison test)\\
    \item Alternating Series
      \subitem \( \sum (-1)^n a_n \)\\
      converges if \( \lim_{n\rightarrow\infty} a_n=0 \) and \( a_{n+1}<a_n \)\\
  \end{itemize}
  \subsection{\color{BrickRed}Convergence Tests}
  \begin{itemize}
    \item $n^{th}$ term test - Only for divergence\\
      \subitem if \( \lim_{n\rightarrow\infty}a_n\neq0 \) then \( \sum
      a_n\) diverges\\
    \item Integral Test\\
      \subitem if \( \int\limits_{1}^{\infty} f(x) \, dx \) converges
      then \( \sum\limits_{n=1}^{\infty} a_n\) converges where \(
      f(x)=a_n\). The converse of this is also true.\\
    \item Limit Comparison\\
      \begin{multicols}{2}
      \subitem if \( \lim_{n\rightarrow\infty} \frac{a_n}{b_n} = L \)
      where L is both positive and finite, then the two series
      both either converge or diverge.\\
      \columnbreak
      \subitem eg:\\
      \( \sum\limits_{n=3}^{\infty} \frac{3}{\sqrt{n^2-4}} \)
      compared to 
      \( \sum\limits_{n=1}^{\infty}\frac{1}{n} \)\\
      \( \lim_{n\rightarrow\infty}\left( \frac{3}{\sqrt{n^2-4}}
      \times \frac{n}{1} \right)=\frac{\infty}{\infty}\)\\
      \(L'H \rightarrow \lim_{n\rightarrow\infty}\left(
      \frac{3\sqrt{n^2-4}}{n} \right)=\frac{\infty}{\infty}\)\\
    \end{multicols}
      \(\lim_{n\rightarrow\infty}\left(
      \frac{3\sqrt{n^2-4}}{\sqrt{n^2}} \right) =
      3\lim_{n\rightarrow\infty}\sqrt{\frac{n^2-4}{n^2}} =
      3\sqrt{\lim_{n\rightarrow\infty}\frac{n^2-4}{n^2}} =
      3 \times 1 = 3 \therefore \sum\limits_{n=0}^{\infty}
      \frac{3}{\sqrt{n^2-4}}\) diverges since \( \frac{1}{n} \)
      diverges and the limit is finite and positive.\\
    \item Ratio Test
      \begin{multicols}{2}
      \subitem \(\lim_{n\rightarrow\infty}\left|
      \frac{a_{n+1}}{a_n} \right|\)\\
      if \(<1 ~ a_n\) converges\\
      if \(>1 ~ a_n\) diverges\\
      if \(=1\) the test is inconclusive\\
      \columnbreak
      \subitem eg:\\
      Find the range of x where
      \(\sum\limits_{n=3}^{\infty}\frac{(-1)^nn!(x-4)^n}{3^n}\)
      converges.\\
      \(\lim_{n\rightarrow\infty}\left|\frac{(n+1)!(x-4)^{n+1}}{3^{n+1}}
      \times \frac{3^n}{n!(x-4)^n}\right|<1\)\\
      \( \lim_{n\rightarrow\infty}\left| \frac{(x-4)(n+1)}{3}
      \right|<1\) \\
      \( \left| \frac{x-4}{3} \right|
      \lim_{n\rightarrow\infty} \left| n+1 \right|<1 \therefore \)
      no range of x makes \(
      \sum\limits_{n=0}^{\infty}\frac{(-1)^nn!(x-4)^n}{3^n}\)
      converge.\\
    \end{multicols}
    \item Condition Convergence\\
      \subitem if \( \sum\limits_{n=1}^{\infty} a_n \) converges
      and \( \sum\limits_{n=1}^{\infty} \left| a_n \right| \)
      diverges\\
  \end{itemize}
  \subsection{\color{BrickRed}Taylor Series}
  \begin{multicols}{2}
  \begin{itemize}
    \item \(
      f(x)=\sum\limits_{n=0}^{\infty}\frac{f^n(c)}{n!}(x-c)^n\)\\
      if \(c=0\) then the series is a MacLaurin series\\
      \subitem
      \(f(x)=\sum\limits_{n=0}^{\infty}\frac{f^n(0)}{n!}x^n\)\\
    \item  \(\sin(x)=\sum\limits_{n=0}^{\infty} (-1)^n
      \frac{x^{2n+1}}{(2n+1)!}\)\\
    \item \(\cos(x)=\sum\limits_{n=0}^{\infty}(-1)^n\frac{x^{2n}}{(2n)!}
     \)\\
    \item\( e^x=\sum\limits_{n=0}^{\infty}\frac{x^n}{n!} \)\\
      \subitem \( e^{-x}=\sum\limits_{n=0}^{\infty}(-1)^n
      \frac{x^n}{n!} \)\\
    \item \(\ln(x+1)=\sum\limits_{n=0}^{\infty}(-1)^n\frac{x^n}{n}\)\\
      \columnbreak
    \item Example of finding the Taylor series centered at 0 (aka
      MacLaurin) of \(e^x\)\\
      \subitem \(f(x)=e^x\)\\
      \subitem \(f(0)=1\)\\
      \(f'(x)=e^x\) and \(f'(0)=1\)\\
      \(f''(x)=e^x\) and \(f''(0)=1\)\\
      \(f'''(x)=e^x\) and \(f'''(0)=1\)\\
      \(\frac{1}{0!}x^0+\frac{1}{1!}x^1+\frac{1}{2!}x^2+\frac{1}{3!}x^3+\cdots+\frac{1}{n!}x^n\)\\
      \(\therefore f(x)=e^x=\sum\limits_{n=0}^{\infty}
      \frac{x^n}{n!}\)\\
  \end{itemize}
\end{multicols}

